% !TEX root = main.tex
%
\section{Markov Decision Processes}
%
\[
\mdp = (\st,\ac,\prob,\rew,\df)
\]
%
%\section{From MDP to LP}
%%
%\subsection{Value LP}
%%
%The primal problem is formulated in the following way.
%%
%\[
%\lpMin{$\sum_{s\in\st} \init_s\cdot v_s$}{
%$v_s \ \geq\ \rew(s,a) + \df \cdot \sum_{s'\in S} \prob_a(s,s')\cdot v_{s'}
%\quad
%\text{for $s\in \st$ and $a\in \ac$}$
%}
%\]
%where $\init : \st \to [0,1]$ is the initial distribution.
%%
%The constraints can also be written as
%\[
%(I_{\ac} - \gamma\prob)\cdot \vec{v} \geq \rew \ .
%\]
%where $I_{\ac}$ is defined as
%\[
%I_{\ac} =
%\begin{drcases}
%\begin{bmatrix}
%\ \id\ \\
%\hline
%\vdots \\
%\hline
%\ \id\ \\
%\end{bmatrix}
%\end{drcases}
%\ac\text{-times}\ ,
%\]
%and $\prob$ is defined as
%\[
%\prob =
%\begin{drcases}
%\begin{bmatrix}
%\vdots\\
%\hline
%\ \prob_a\ \\
%\hline
%\vdots \\
%\end{bmatrix}
%\end{drcases}
%a\in\ac\ .
%\]
%\subsection{Policy LP}
%%
%The dual LP becomes
%\[
%\lpMax{$\sum_{s\in\st}\sum_{a\in\ac} \rew(s,a)\cdot x_{(s,a)}$}{
%  $\transpose{(I_{\ac} - \df\prob)}\cdot\vec{x} = \init$ \\
%& $\vec{x} \geq \vec{0}$ \ .
%}
%\]
%%
%Expanding using the definitions of $\ac$ an $\prob$ and doing a slight rearrangement of the terms,
%we can see that the set of constraints for the dual LP contains,
%for every $s\in\st$ the equation
%\[
%\sum_{a\in\ac} x_{s,a} \ -\ \df\cdot\sum_{s'\in S} \prob_a(s',s)\cdot x_{s',a}
%\ = \ \init_{s} \ .
%\]
%%
%The variables $x_{(s,a)}$ can be interpreted as expected discounted number of times the action $a\in\ac$ is chosen from the state $s\in S$.
%\section{Newton's method}
%%
%We follow \cite{vishnoi}.
%We use Newton's method on the dual system.
%Since its system of constraitns contains a system of equations along with non-negativity constraints we have to use \cite[Section 11.2]{vishnoi}.
%
%Since $\df < 1$, the matrix $\id_{\st} - \df\cdot \prob_{a}$ is invertible for any $a\in\ac$,
%hence we do not need to delete any rows.
%
%Let $A = \transpose{(I_{\ac} - \lambda\cdot\prob)}$.
%Using \cite[equation 11.8]{vishnoi} we get that the Newton step is:
%\[
%\step{\eta}(\vec{x}) = X^2 (\transpose{A}(A X^{-2} \transpose{A})^{-1} A X^2 - \id_{\st})(\eta\cdot \rew + X^{-1}\cdot \vec{1})
%\]
%Here $X$ is the diagonal matrix with $x_{s,a}$ as the diagonal entries.
%
\section{Partially observable MDPs}
%
\[
\pdp = (\mdp,\sig,\obpr)
\]
%
For the remainder of the article, fix a POMDP $\pdp$. 
From $\pdp$, we can construct a (possibly) non-deterministic finite automata \emphdef{$\aut{\pdp}$} on the alphabet $\ac \times \sig$ with $\st$ as the set of states and the transition relation $\Delta$ defined as
\[
(p,(a,\alpha),q) \in \Delta
\]
if there is a non-zero probabilty of going to the state $q$ from the state $p$ after taking the action $a$ and seeing the observation $\alpha$. 
%
\begin{definition}
The POMDP $\pdp$ is called \emphdef{deterministic} if $\aut{\pdp}$ is deterministic.

In this case we use $\delta$ to denote the transition \emph{function} of $\aut{\pdp}$.
\end{definition}
%
\begin{definition}
A \emphdef{belief of $\pdp$} is a distribution over $\st$.
\end{definition}
%
Beliefs are denoted as boldface letters $\bel{a},\bel{b},\dots$.
The space of all beliefs of $\pdp$ is denote as \emphdef{$\bsp{\pdp}$}.
The \emphdef{support} of a belief $\bel{b}$ is defined as
\[
\emphdef{$\supp{\bel{b}}$} = 
\setof{q\in\st}{\bel{b}(q)\neq 0} \ .
\]
For $n\in\N$, we use \emphdef{$\bsup[n]{\pdp}$} to denote the set of all belief supports of size $n$.
Note that $\bsup[n]{\pdp} = \emptyset$ when $n > |\st|$.
We define
\[
\emphdef{$\bsup{\pdp}$} = \bigcup_{n\in\N} \bsup[n]{\pdp} \ .
\]
%
Extending the transition function $\delta$ of $\aut{\pdp}$ to $\bsup{\pdp}$ as
\[
\delta(S,(a,\alpha)) =
\setof{\delta(p,(a,\alpha))}{p\in S}
\]
we get a deterministic finite automata \emphdef{$\baut{\pdp}$} with set of states $\bsup{\pdp}$ and alphabet $\ac\times\sig$.

Whenever convenient, we consider $\baut{\pdp}$ to be a directed graph with edges labelled by elements from $\ac\times\sig$.
%